%-------------------------------------------------------------------------------
% Document: OpenCTM Format Specification
% Author:   Marcus Geelnard
% Compile:  pdflatex FormatSpecification.tex
%-------------------------------------------------------------------------------
% Note: You need a LaTeX environment to build this document as PDF. The
% recommended way is to install TeX Live (http://www.tug.org/texlive/) and
% a decent LaTeX editor (e.g. texmaker, LEd, etc).
%
% Ubuntu:   sudo apt-get install texlive-full
% Mac OS X: http://www.tug.org/mactex/  (MacTeX.mpkg.zip)
%
% To build the PDF document, run pdflatex twice on this .tex file (in order to
% correctly build the TOC).
%-------------------------------------------------------------------------------

% Use the OpenCTM TeX style
\input{openctm-tex.sty}

% Document properties
\author{Marcus Geelnard}
\title{OpenCTM Format Specification}

% Document contents
\begin{document}

%--[ Title page ]---------------------------------------------------------------

\begin{titlepage}
\begin{center}
~
\vspace{5cm}

\includegraphics[width=10.0cm]{logo.pdf}
\vspace{0.4cm}

{\large File format version 5}

\vspace{1.0cm}

{\Large Format Specification}
\vspace{1.5cm}

Copyright \copyright \ 2009 Marcus Geelnard
\end{center}
\end{titlepage}


%--[ Table of contents ]--------------------------------------------------------

\tableofcontents


%-------------------------------------------------------------------------------

\chapter{Introduction}
This document describes version 5 of the OpenCTM file format.


%-------------------------------------------------------------------------------

\chapter{File format specification}
The structure of a OpenCTM file is as follows:

[Header]\newline 
[Body data]

Each part of the file is described in the following sections.

\section{Data formats}
All integer fields are stored in 32-bit little endian format (least significant
byte first).

All floating point fields are stored in 32-bit binary IEEE 754 format (little
endian).

All strings are stored as a 32-bit integer string length (number of bytes)
followed by a UTF-8 format string (there is no zero termination and no BOM).


\section{Header}
The file must start with a header, which looks as follows:

\begin{tabular}{|l|l|l|}\hline
\textbf{Offset} &  \textbf{Type} & \textbf{Description}\\ \hline
0 & Integer & Magic identifier (0x4d54434f, or "OCTM" when read as ASCII).\\ \hline
4 & Integer & File format version (0x00000005 = version 5).\\ \hline
8 & Integer & Compression method, which can be one of the following:\\
 & & 0x00574152 - Use the RAW compression method.\\
 & & 0x0031474d - Use the MG1 compression method.\\
 & & 0x0032474d - Use the MG2 compression method.\\ \hline
12 & Integer & Vertex count.\\ \hline
16 & Integer & Triangle count.\\ \hline
20 & Integer & UV map count.\\ \hline
24 & Integer & Attribute map count.\\ \hline
28 & Integer & Boolean flags, or:ed together:\\
 & & 0x00000001 - The file contains per-vertex normals.\\ \hline
32 & String & File comment.\\ \hline
\end{tabular}

The length of the file header is varibale, since it contains fields of variable
length.


\section{Body data}
The body data follows immediately after the file header. Its file offset is
dictated by the length of the file header.

The format of the body data is specific for each compression method, which is
defined by the "Compression method" field in the header.

The body data contains the vertex, index, normal, UV map and attribute map
data, usually in a compressed form.


\subsection{RAW}
To be written...

Please read the source code file compressRAW.c for more details on the body
format when the RAW compression method is used.


\subsection{MG1}
To be written...

Please read the source code file compressMG1.c for more details on the body
format when the MG1 compression method is used.


\subsection{MG2}
To be written...

Please read the source code file compressMG2.c for more details on the body
format when the MG2 compression method is used.

\end{document}
