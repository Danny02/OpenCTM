%-------------------------------------------------------------------------------
% Document: OpenCTM Developers Manual
% Author:   Marcus Geelnard
% Compile:  pdflatex DevelopersManual.tex
%-------------------------------------------------------------------------------
% Note: You need a LaTeX environment to build this document as PDF. The
% recommended way is to install TeX Live (http://www.tug.org/texlive/) and
% a decent LaTeX editor (e.g. texmaker, LEd, etc).
%
% Ubuntu:   sudo apt-get install texlive-full
% Mac OS X: http://www.tug.org/mactex/  (MacTeX.mpkg.zip)
%
% To build the PDF document, run pdflatex twice on this .tex file (in order to
% correctly build the TOC).
%-------------------------------------------------------------------------------

% Use the OpenCTM TeX style
\input{openctm-tex.sty}

% Document properties
\author{Marcus Geelnard}
\title{OpenCTM Developers Manual}

% Document contents
\begin{document}

%--[ Title page ]---------------------------------------------------------------

\begin{titlepage}
\begin{center}
~
\vspace{5cm}

\includegraphics[width=10.0cm]{logo.pdf}
\vspace{0.4cm}

{\large Software Library version 0.7 (beta)}

\vspace{1.0cm}

{\Large Developers Manual}
\vspace{1.5cm}

Copyright \copyright \ 2009 Marcus Geelnard
\end{center}
\end{titlepage}


%--[ Table of contents ]--------------------------------------------------------

\tableofcontents


%-------------------------------------------------------------------------------

\chapter{Introduction}
The OpenCTM file format is an open format for storing 3D triangle meshes.
One of the main advantages over other similar file formats is its ability
to losslessly compress the triangle geometry to a fraction of the corresponding
raw data size.

This document describes how to use the OpenCTM API to load and save OpenCTM
format files. It is mostly written for C/C++ users, but should be useful for
other programming languages too, since the concepts and function calls are
virtually identical regardless of programming language.


%-------------------------------------------------------------------------------

\chapter{Concepts}

\section{The OpenCTM API}
The OpenCTM API makes it easy to read and write OpenCTM format files. The API is
implemented in the form of a software library that an application can be linked
to in order to access the OpenCTM API.

The software library itself is written in standard, portable C language, but
can be used from many other programming languages (writing language bindings
for new languages should be fairly straight forward, since the API was written
with cross-language portability in mind).

The full documentation of the OpenCTM API can be found in the Doxygen-generated
documentation ("API Reference"), which describes all API functions, types,
constants etc.


\section{The triangle mesh}
The triangle mesh, in OpenCTM terms, is managed in a format that is well suited
for a modern 3D rendering pipeline, such as OpenGL.

At a glance, the OpenCTM mesh has the following properties:

\begin{itemize}
    \item A vertex is a set of attributes that uniquely identify the vertex.
          This includes: vertex coordinate, normal, texture coordinate(s) and
          custom vertex attribute(s) (such as color, weight, etc).
    \item A triangle is described by three vertex indices.
    \item In the OpenCTM API, these mesh data are treated as arrays (an integer
          array for the triangle indices, and floating point arrays for the
          vertex data).
    \item All vertex data arrays in a mesh must have the same number of elements
          (for instance, there is exactly one normal associated with each
          vertex coordinate).
    \item All mesh data are optional, except for the triangle indices and the
          vertex coordinates. For instance, it is possible to leave out the
          normal information.
\end{itemize}

For an example of the mesh data structure see table \ref{tab:MeshVert} (vertex
data) and table \ref{tab:MeshTri} (triangle data).

\begin{table}[p]
\centering
\begin{tabular}{|l|l|l|l|l|l|l|l|}\hline
\textbf{Index} & 0 & 1 & 2 & 3 & 4 & \textellipsis & N\\ \hline
\textbf{Vertex} & $v_0$ & $v_1$ & $v_2$ & $v_3$ & $v_4$ & \textellipsis & $v_N$\\ \hline
\textbf{Normal} & $n_0$ & $n_1$ & $n_2$ & $n_3$ & $n_4$ & \textellipsis & $n_N$\\ \hline
\textbf{TexCoord1} & $t1_0$ & $t1_1$ & $t1_2$ & $t1_3$ & $t1_4$ & \textellipsis & $t1_N$\\ \hline
\textbf{TexCoord2} & $t2_0$ & $t2_1$ & $t2_2$ & $t2_3$ & $t2_4$ & \textellipsis & $t2_N$\\ \hline
\textbf{Attrib1} & $a1_0$ & $a1_1$ & $a1_2$ & $a1_3$ & $a1_4$ & \textellipsis & $a1_N$\\ \hline
\textbf{Attrib2} & $a2_0$ & $a2_1$ & $a2_2$ & $a2_3$ & $a2_4$ & \textellipsis & $a2_N$\\ \hline
\end{tabular}
\caption{Mesh vertex data structure in OpenCTM, for a mesh with normals,
two texture coordinates per vertex, and two custom attributes per vertex.}
\label{tab:MeshVert}
\end{table}

\begin{table}[p]
\centering
\begin{tabular}{|l|l|l|l|l|l|l|l|}\hline
\textbf{Triangle} & $tri_0$ & $tri_1$ & $tri_2$ & $tri_3$ & $tri_4$ & \textellipsis & $tri_M$\\ \hline
\end{tabular}
\caption{Mesh triangle data structure in OpenCTM, where $tri_k$ is a tuple of
three vertex indices. For instance,
$tri_0=(0, 1, 2)$,
$tri_1=(0, 2, 3)$,
$tri_2=(3, 5, 4)$, \textellipsis}
\label{tab:MeshTri}
\end{table}


\subsection{Triangle indices}
\label{sec:MeshIndices}

Each triangle is described by three integers: one vertex index for each corner
of the triangle). The triangle index array looks like this:

\begin{tabular}{|l|l|l|l|l|l|l|l|l|l|}\hline
$tri^0_0$ & $tri^1_0$ & $tri^2_0$ & $tri^0_1$ & $tri^1_1$ & $tri^2_1$ & \textellipsis & $tri^0_M$ & $tri^1_M$ & $tri^2_M$\\ \hline
\end{tabular}

\textellipsis where $tri^j_k$ is the vertex index for the $j$:th corner of the
$k$:th triangle.

It is recommended (but not required) that:
\begin{itemize}
    \item \textellipsis given the three coordinates of a triangle,
          $p_1, p_2, p_3$, the front/out direction of the triangle (the "flat
          normal") should be defined by the normalized cross product:
          $(p_2-p_1)\times (p_3-p_1)$.
\end{itemize}


\subsection{Vertex coordinates}

Each vertex coordinate is described by three floating point values: $x$, $y$
and $z$. The vertex coordinate array looks like this:

\begin{tabular}{|l|l|l|l|l|l|l|l|l|l|l|}\hline
$x_0$ & $y_0$ & $z_0$ & $x_1$ & $y_1$ & $z_1$ & \textellipsis & $x_N$ & $y_N$ & $z_N$\\ \hline
\end{tabular}

\textellipsis where $x_k$, $y_k$ and $z_k$ are the $x$, $y$ and $z$ coordinates
of the $k$:th vertex.

It is recommended (but not required) that:
\begin{itemize}
    \item \textellipsis the "up" direction of all vertex coordinates is the
          positive $z$-axis.
    \item \textellipsis the unit of all vertex coordinates is meters ($m$).
\end{itemize}


\subsection{Normals}

Each normal is described by three floating point values: $x$, $y$
and $z$. The normal array looks like this:

\begin{tabular}{|l|l|l|l|l|l|l|l|l|l|l|}\hline
$x_0$ & $y_0$ & $z_0$ & $x_1$ & $y_1$ & $z_1$ & \textellipsis & $x_N$ & $y_N$ & $z_N$\\ \hline
\end{tabular}

\textellipsis where $x_k$, $y_k$ and $z_k$ are the $x$, $y$ and $z$ components
of the $k$:th normal.

It is recommended (but not required) that:
\begin{itemize}
    \item \textellipsis all normals have unit lengths (i.e.
          $x_k^2+y_k^2+z_k^2=1\:\forall\:k$).
    \item \textellipsis the vertex normal points in the "out" direction of the
          surface, as defined by the flat normals of the neighboring triangles
          (see \ref{sec:MeshIndices}).
\end{itemize}


\subsection{Texture coordinates}

A mesh may have several texture maps, where each texture map is described by:

\begin{itemize}
    \item A texture coordinate array.
    \item A unique texture map name.
    \item An file name reference (optional).
\end{itemize}

Each texture coordinate is described by two floating point values: $u$ and $v$.
A texture coordinate array looks like this:

\begin{tabular}{|l|l|l|l|l|l|l|l|l|l|}\hline
$u_0$ & $v_0$ & $u_1$ & $v_1$ & $u_2$ & $v_2$ & \textellipsis & $u_N$ & $v_N$\\ \hline
\end{tabular}

\textellipsis where $u_k$ and $v_k$ are the $u$ and $v$ components
of the $k$:th texture coordinate.

$(u=0, v=0)$ represents the lower left corner of the texture, and $(u=1, v=1)$
represents the upper right corner of the texture.

It is recommended (but not required) that:
\begin{itemize}
    \item \textellipsis the texture coordinates are in the range $[0,1]$.
\end{itemize}


\subsection{Custom vertex attributes}

A mesh may have several custom vertex attribute maps, where each attribute map
is described by:

\begin{itemize}
    \item A vertex attribute array.
    \item A unique attribute map name.
\end{itemize}

Each vertex attribute is described by four floating point values: $a$, $b$, $c$
and $d$. An attribute array looks like this:

\begin{tabular}{|l|l|l|l|l|l|l|l|l|l|l|l|l|}\hline
$a_0$ & $b_0$ & $c_0$ & $d_0$ & $a_1$ & $b_1$ & $c_1$ & $d_1$ & \textellipsis & $a_N$ & $b_N$ & $c_N$ & $d_N$\\ \hline
\end{tabular}

\textellipsis where $a_k$, $b_k$, $c_k$ and $d_k$ are the four attribute values
of the $k$:th attribute.

It is recommended (but not required) that:
\begin{itemize}
    \item \textellipsis the value range of clamped attributes, such as color
          or percentage attributes, is $[0,1]$.
    \item \textellipsis unused components are set to either zero or one,
          depending on the context.
    \item \textellipsis the attribute map name is a descriptive name of the
          attribute.
\end{itemize}

The following attribute names and formats should be used whenever possible to
represent specific information (please note that names are case sensitive):

\begin{tabular}{|l|l|l|l|}\hline
\textbf{Name} & \textbf{Format $(a,b,c,d)$} & \textbf{Range} & \textbf{Description}\\ \hline
Color & $(R,G,B,A)$ & $[0,1]$ & Color and alpha (opacity).\\ \hline
Weight & $(W,0,0,0)$ & $[0,1]$ & Weight factor.\\ \hline
\end{tabular}


\section{The OpenCTM context}
The OpenCTM API uses a \emph{context} for almost all operations (function calls).
The context is created and destroyed with the functions ctmNewContext() and
ctmFreeContext(), respectively.

A program may instantiate any number of contexts, and all OpenCTM function
calls are completely thread safe (multiple threads can use the OpenCTM API
at the same time), as long as each context instance is handled by a single
thread.

Each context is fully self contained and independent of other contexts.

There are two types of OpenCTM context: \emph{import contexts} and
\emph{export contexts}. Import contexts are used for importing OpenCTM files,
and export contexts are used for exporting OpenCTM files.

The context type is selected when creating the context.



%-------------------------------------------------------------------------------

\chapter{Compression Methods}
The OpenCTM file format supports a few different compression methods, each
with its own advantages and disadvantages. The API makes it possible to
select which method to use when creating OpenCTM files (the default method
is MG1).


\section{RAW}
The RAW compression method is not really a compression method, since it only
stores the data in a raw, uncompressed form. The result is a file with the same
size and data format as the in-memory mesh data structure.

The RAW method is mostly useful for testing purposes, but can be preferred in
certain situations, for instance when file writing speeds and a small memory
footprint is more important than minimizing file sizes.

Another situation where the RAW method can be useful is when you need an
easily parsable binary file format. Usually the OpenCTM API can be used in
almost any application, but in some environments, such as certain script
languages or data inspecion tools, it can be handy to have access to the
raw data.


\section{MG1}
The MG1 compression method effectively reduces the size of the mesh data
by re-coding the connectivity information of the mesh into an easily
compressible format. The data is then compressed using LZMA.

The floating point data, such as vertex coordinates and normals, is fully
preserved in the MG1 method, by simply applying lossless LZMA compression
to it. 

Under typical condititions, the connectivity information is compressed to
about two bytes per triangle (17\% of the original size), and vertex data
is compressed to about 75\% of the original size.

While creating MG1 files can be a relatively slow process (compared to the
RAW method, for instance) the reading speed is usually very high, thanks to
the fast LZMA decoder and the uncomplicated data format.


\section{MG2}
The MG2 compression method offers the highest level of compression among the
different OpenCTM methods. It uses the same method for compressing connectivity
information as the MG1 method, but does a better job at compressing vertex
data.

Vertex data is converted to a fixed point representation, which allows for
efficient, lossless, prediction based data compression algorithms.

In short, the MG2 method divides the mesh into small sub-spaces, sorts the data
geometrically, and applies delta-prediction to the data, which effectively
lowers the data entropy. The re-coded vertex data is then compressed with
LZMA.

When using the OpenCTM API for creating MG2 files you can trade mesh resolution
for compression ratio, and the API provides several functions for controlling
the resolution of different vertex attributes independently. Therefor it is
usually important to know the resolution requirements for your specific
application when using the MG2 method.

In some applications, such as games, movies and art, it is important that the
3D model is not visually degraded by compression. In such applications
you will typically tune your resolution settings using trial and error,
until you find a setting that does not alter the model visually.

In other applications, such as CAD/CAM, 3D scanning, calibration, etc,
reasonable resolution settings can usually be derived from the limitations
of the process in which the model is used. For instance, there is usually no
need for nanometer precision in the design of an airplane wing, and there
is little use of micrometer resolution in a manufacturing process that can
not reproduce features smaller than 0.15 mm.

As a side effect of the fact that MG2 produces smaller files than the MG1
method does, loading files is usually faster with the MG2 method than with
the MG1 method. Saving files with the MG2 method is about as fast as with
the MG1 method.



%-------------------------------------------------------------------------------

\chapter{Basic Usage}

\section{Prerequisites}
To use the OpenCTM API, you need to include the OpenCTM include file, like this:

\begin{lstlisting}
#include <openctm.h>
\end{lstlisting}

You also need to link with the OpenCTM import library. For instance, in MS
Visual Studio you can add "openctm.lib" to your Additional Dependencies field
in the Linker section. For gcc/g++ or similar compilers, you will typically
add -lopenctm to the list of compiler options, for instance:

\begin{lstlisting}
> g++ -o foo foo.cpp -lopenctm
\end{lstlisting}


\section{Loading OpenCTM files}
Below is a minimal example of how to load an OpenCTM file with the OpenCTM API,
in just a few lines of code:

\begin{lstlisting}
CTMcontext context;
CTMuint vertCount, triCount, * indices;
CTMfloat * vertices;

// Create a new context
context = ctmNewContext(CTM_IMPORT);

// Load the OpenCTM file
ctmLoad(context, "mymesh.ctm");
if(ctmGetError(context) == CTM_NONE)
{
  // Access the mesh data
  vertCount = ctmGetInteger(context, CTM_VERTEX_COUNT);
  vertices = ctmGetFloatArray(context, CTM_VERTICES);
  triCount = ctmGetInteger(context, CTM_TRIANGLE_COUNT);
  indices = ctmGetIntegerArray(context, CTM_INDICES);

  // Deal with the mesh (e.g. transcode it to our
  // internal representation)
  // ...
}

// Free the context (this frees all memory allocated by
// the OpenCTM context)
ctmFreeContext(context);
\end{lstlisting}


\section{Creating OpenCTM files}
To be written...



%-------------------------------------------------------------------------------

\chapter{Controlling Compression}
To be written...


\section{Selecting the compression method}
To be written...


\section{Selecting fixed point precision}
To be written...



%-------------------------------------------------------------------------------

\chapter{Error Handling}
An error can occur when calling any of the OpenCTM API functions. To check
for errors, call the ctmGetError() function, which returns a positive error
code if something went wrong, or zero (CTM\_NONE) if no error has occured.

See \ref{tab:ErrorCodes} for a list of possible error codes.

\begin{table}[p]
\centering
\begin{tabular}{|l|p{7cm}|}\hline
\textbf{Code} & \textbf{Description}\\ \hline
CTM\_NONE (zero) & No error has occured (everything is OK).\\ \hline
CTM\_INVALID\_CONTEXT & The OpenCTM context was invalid (e.g. NULL).\\ \hline
CTM\_INVALID\_ARGUMENT & A function argument was invalid.\\ \hline
CTM\_INVALID\_OPERATION & The operation is not allowed.\\ \hline
CTM\_INVALID\_MESH & The mesh was invalid (e.g. no vertices).\\ \hline
CTM\_OUT\_OF\_MEMORY & Not enough memory to proceed.\\ \hline
CTM\_FILE\_ERROR & File I/O error.\\ \hline
CTM\_BAD\_FORMAT & File format error (e.g. unrecognized format or corrupted file).\\ \hline
CTM\_LZMA\_ERROR & An error occured within the LZMA library.\\ \hline
CTM\_INTERNAL\_ERROR & An internal error occured (indicates a bug).\\ \hline
CTM\_UNSUPPORTED\_FORMAT\_VERSION & Unsupported file format version.\\ \hline
\end{tabular}
\caption{OpenCTM error codes.}
\label{tab:ErrorCodes}
\end{table}

The last error code that indicates a failure is stored per OpenCTM context
until the ctmGetError() function is called. Calling the function will reset
the error state.

It is also possible to convert an error code to a an error string, using the
ctmErrorString() function, which takes an error code as its argument, and
returns a constant C string (pointer to a null terminated UTF-8 format
character string).



%-------------------------------------------------------------------------------

\chapter{C++ Extensions}
To take better advantage of some of the C++ language features, such as
exception handling, a few C++ wrapper classes are availbale through the standard
API when compiling a C++ program. As usual, just include "openctm.h", and you
will have access to two C++ classes: CTMimporer and CTMexporter.

The main differences between the C++ classes and the standard API are:

\begin{itemize}
    \item The C++ classes call ctmNewContext() and ctmFreeContext() in their
          constructors and destructors respectively, which makes it easier to
          use the C++ dynamic scope mechanisms (such as exception handling).
    \item Whenever an OpenCTM error occurs, an exception is thrown. Hence, there
          is no method corresponding to the ctmGetError() function.
\end{itemize}

\section{The CTMimporter class}

Here is an example of how to use the CTMimporter class in C++:

\begin{lstlisting}
try
{
  // Create a new OpenCTM importer object
  CTMimporter ctm;

  // Load the OpenCTM file
  ctm.Load("mymesh.ctm");

  // Access the mesh data
  CTMuint vertCount = ctm.GetInteger(CTM_VERTEX_COUNT);
  CTMfloat * vertices = ctm.GetFloatArray(CTM_VERTICES);
  CTMuint triCount = ctm.GetInteger(CTM_TRIANGLE_COUNT);
  CTMuint * indices = ctm.GetIntegerArray(CTM_INDICES);

  // Deal with the mesh (e.g. transcode it to our
  // internal representation)
  // ...
}
catch(exception &e)
{
  cout << "Error: " << e.what() << endl;
}
\end{lstlisting}


\section{The CTMexporter class}
To be written...


\end{document}
